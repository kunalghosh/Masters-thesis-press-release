\documentclass[a4paper]{article}
\usepackage[left=3cm,right=3cm,top=3cm,bottom=3cm,footnotesep=1cm]{geometry}
\usepackage{mathpazo}
\usepackage{ragged2e}

%opening
%\title{\justify\fontsize{20}{20}\selectfont{Aalto University researchers predict molecular properties using AI, this could significantly speed up the search for new materials in the future.}}
%\title{\justify\fontsize{20}{20}\selectfont{Search for new materials could be significantly faster in the future, using the AI developed by a group of researchers at Aalto University.}}
\title{\justify\fontsize{20}{20}\selectfont{AI developed by a group of researchers at Aalto University could significantly speed up the search for new materials.}}

\begin{document}

\date{}
\maketitle
% \begin{abstract}
%     
%     % Complex quantum mechanics governing the properties of materials, like how well a solar cell can generate electricity,
%     % were previously simulated using 
% \end{abstract}

\noindent \textbf{Espoo, Finland - October 5, 2017:} New materials which are designed specifically for a particular task, such as materials for mobile phone batteries which can hold charge for a long time, are everywhere around us. A group of researchers from the machine learning and applied physics departments at Aalto University, School of Science, have used an artificial intelligence (AI) powered by deep neural networks to predict molecular properties directly from the shape and atomic composition of molecules. The AI can predict these properties, for individual molecules, in milliseconds compared to minutes taken by traditional techniques. The AI has also achieved state-of-the-art results in predicting the molecular orbital\footnote{Specifically, the sixteen highest occupied molecular energies (HOMO)} energies and is the first\footnote{First AI to make spectra predictions.} to predict molecular spectra.\\

When searching for new materials, scientists usually start with a large database of potential molecules. A crucial first step is to study each molecule individually looking for properties favourable for the intended application of the final material. The molecular spectra help scientists identify these properties and conventionally, computer simulations have been used to compute them. Since the simulations are repeated for each molecule the traditional methods are slow. The AI, on the contrary, uses the spectra using a large database of pre-computed, molecule-spectra pairs. Although these molecule-spectra pairs are computed using traditional techniques, and the learning of the AI is slow, the predictions of spectra for new molecules can be made instantly. Since the learning and data generation is only done once, the AI is much faster overall.\\

In the thesis, we compared three machine learning algorithms, including hte previous state-of-the-art, and found our proposed algorithm to perform the best when predicting the molecular orbital energies. This was the benchmark used by researchers previously and is closely related to the molecular spectra discussed earlier.\\



% \justify\large{Deep neural networks (DNNs) were used, for the first time, to predict molecular photo-emission spectra 

% These methods also achieved state-of-the-art results in predicting molecular orbital energies. DNN predictions

% take only a few milliseconds compared to hours taken by traditional techniques.}

% Researchers from the machine learning and applied physicis departments at Aalto university school of science collaborated on the project to predict molecular properties. When searching for novel materials, for example a battery that can double the run-time of mobile phonesdirectly from the geometry of molecules, a crucial step is to first analyse properties of molecules looking for favourable features. Different forms of spectrums help scientists identify these features.
 
% Every time we use the non-stick pan in the kitchen or ride electric cars, we are direct benefactors

% of applications of novel materials. Teflon in the case of non-stick pans and high-efficiency lithium polymer batteries in cars.


\noindent \textbf{For further information, please contact:}\\

\noindent Kunal Ghosh\\
Dept. of Computer Science.\\
Aalto University, Finland.\\
kunal.ghosh(at)aalto.fi\\



\end{document}
